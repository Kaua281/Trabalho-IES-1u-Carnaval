\documentclass[12pt,oneside,a4paper,article]{abntex2}
\usepackage[utf8]{inputenc} % Codificação do documento
\usepackage[T1]{fontenc}    % Seleção de código de fonte.
\usepackage[brazil]{babel}  % Idioma do documento
\usepackage{graphicx}       % Inclusão de gráficos
\usepackage{tabularx}       % Tabelas avançadas
\usepackage{amsmath}        % Melhorias em matemática
\usepackage{lipsum}         % Geração de texto dummy
\usepackage{authblk}

% Configurações específicas do abntex2
% Aqui você pode adicionar configurações específicas, como redefinições de comandos
% ou adições de novos pacotes que são essenciais para o seu documento.

% Carrega o pacote abntex2cite para citações
\usepackage[alf]{abntex2cite} % ou use [num] para citações numéricas

\usepackage[left=3cm,right=2cm,top=3cm,bottom=2cm]{geometry} % Margens
\usepackage{setspace}       % Espaçamento entre linhas
% %\usepackage{natbib}         % Formatação de bibliografia

% Informações de título
\title{\textbf{Seu Título Aqui}}
\author{Kauã Oliveira Seixas \thanks{kaua.seixas@ucsal.edu.br}}
\author{Rivaldo de Jesus Santos \thanks{rivaldo.santos@ucsal.edu.br}}
\author[1]{Vinícius Scola Santana \thanks{viniciusscola.santana@ucsal.edu.br}}
\author[1]{Eduardo Campos Aguiar \thanks{eduardo.aguiar@ucsal.edu.br}}
\author[1]{Cauã César Rodrigues Costa\thanks{cauacesar.costa@ucsal.edu.br} }
\author[1*]{Orientador: Elton Figueiredo da Silva \thanks{elton.figueiredo@pro.ucsal.br}}


\pagenumbering{gobble} % Remove numeração de página

\affil{
  Bacharelado em Engenharia de Software \par
  Escola de Tecnologias \par
Universidade Católica do Salvador (UCSAL) \par
Av. Prof. Pinto de Aguiar, 2589 Pituaçu, CEP: 41740-090 \par
Salvador/BA, Brasil
}
% !TODO - Verificar com Professor se é essa formatação mesmo, ou é algum erro no modelo
\affil[1]{\textit {\{Kauã Oliveira Seixas, Rivaldo de Jesus Santos, Vinícius Scola Santana
, Eduardo Campos Aguiar, Cauã César Rodrigues Costa\}@ucsal.edu.br}}
\affil[1*]{\textit {\{Elton Figueiredo da Silva\}@pro.ucsal.edu.br}}




\date{Abril 2025}



\ifthenelse{\equal{\ABNTEXisarticle}{true}}{%
\renewcommand{\maketitlehookb}{}
}{}

% Configurações de aparência do PDF final
% \usepackage{hyperref} % para inserir links
 \hypersetup{
      colorlinks=false,       % false: boxed links; true: colored links
      pdfborder={0 0 0},      % remove as bordas ao redor dos links
 }

\renewcommand*{\Authsep}{, }
\renewcommand*{\Authand}{, }
\renewcommand*{\Authands}{, }
\renewcommand*{\Affilfont}{\normalsize\normalfont}
\renewcommand*{\Authfont}{\bfseries}    % make author names boldface    
\setlength{\affilsep}{2em}   % set the space between author and affiliation

\newsavebox\affbox


\begin{document}

\begin{center}
  \includegraphics[width=0.3\textwidth]{imagens-template/ucsal_logo.png}
\end{center}
{\let\newpage\relax\maketitle}

\clearpage
\pagenumbering{arabic} % Retoma a numeração normal
\begin{resumoumacoluna}
  \lipsum[1] % Gera um texto de exemplo
  \vspace{\onelineskip}

  \noindent
  \textbf{Palavras-chaves}: palavra1, palavra2, palavra3.
\end{resumoumacoluna}

\clearpage

\textual

% Apresentação do Tema, Contextualização 
\section{Introdução}
A transição do mundo analógico para o digital fez com que a sociedade incorporasse o uso de tecnologias em seu cotidiano, inclusive para acessar serviços governamentais. Diante disso, o Estado busca adaptar-se a essa realidade, promovendo a digitalização e a simplificação do acesso aos seus serviços \cite{viana2021transformaccao}.
A Engenharia de Software é uma profissão que tem um papel fundamental nesse contexto, pois é catalisadora da transformação digital, portanto ela é o agente responsável pela adaptação dos serviços governamentais para o mundo digital.


\section{Desenvolvimento}
\lipsum[5-6]
\citeonline{ribeiro:100}

\subsection{Subseção de Exemplo}
\lipsum[7]

\section{Conclusão}
\lipsum[8-9]



% Formatação da bibliografia
%bibliographystyle{plain}
\bibliography{referencias} % Assume que você tem um arquivo referencias.bib

\end{document}
