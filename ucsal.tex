\documentclass[12pt,oneside,a4paper,article]{abntex2}
\usepackage[utf8]{inputenc} % Codificação do documento
\usepackage[T1]{fontenc}    % Seleção de código de fonte.
\usepackage[brazil]{babel}  % Idioma do documento
\usepackage{graphicx}       % Inclusão de gráficos
\usepackage{tabularx}       % Tabelas avançadas
\usepackage{amsmath}        % Melhorias em matemática
\usepackage{lipsum}         % Geração de texto dummy
\usepackage{authblk}

% Configurações específicas do abntex2
% Aqui você pode adicionar configurações específicas, como redefinições de comandos
% ou adições de novos pacotes que são essenciais para o seu documento.

% Carrega o pacote abntex2cite para citações
\usepackage[alf]{abntex2cite} % ou use [num] para citações numéricas

\usepackage[left=3cm,right=2cm,top=3cm,bottom=2cm]{geometry} % Margens
\usepackage{setspace}       % Espaçamento entre linhas
% %\usepackage{natbib}         % Formatação de bibliografia

% Informações de título
\title{\textbf{A Digitalização dos Serviços Públicos: O Papel da Engenharia de Software na Inovação Governamental}}
\author{Kauã Oliveira Seixas \thanks{kaua.seixas@ucsal.edu.br}}
\author{Rivaldo de Jesus Santos \thanks{rivaldo.santos@ucsal.edu.br}}
\author[1]{Vinícius Scola Santana \thanks{viniciusscola.santana@ucsal.edu.br}}
\author[1]{Eduardo Campos Aguiar \thanks{eduardo.aguiar@ucsal.edu.br}}
\author[1]{Cauã César Rodrigues Costa\thanks{cauacesar.costa@ucsal.edu.br} }
\author[1*]{Orientador: Elton Figueiredo da Silva \thanks{elton.figueiredo@pro.ucsal.br}}


\pagenumbering{gobble} % Remove numeração de página

\affil{
  Bacharelado em Engenharia de Software \par
  Escola de Tecnologias \par
Universidade Católica do Salvador (UCSAL) \par
Av. Prof. Pinto de Aguiar, 2589 Pituaçu, CEP: 41740-090 \par
Salvador/BA, Brasil
}

\affil[1]{\textit {\{kaua.seixas, rivaldo.santos, viniciusscola.santana
, eduardo.aguiar, cauacesar.costa\}@ucsal.edu.br}}
\affil[1*]{\textit {\{elton.figueiredo\}@pro.ucsal.edu.br}}




\date{Março 2025}



\ifthenelse{\equal{\ABNTEXisarticle}{true}}{%
\renewcommand{\maketitlehookb}{}
}{}

% Configurações de aparência do PDF final
% \usepackage{hyperref} % para inserir links
 \hypersetup{
      colorlinks=false,       % false: boxed links; true: colored links
      pdfborder={0 0 0},      % remove as bordas ao redor dos links
 }

\renewcommand*{\Authsep}{, }
\renewcommand*{\Authand}{, }
\renewcommand*{\Authands}{, }
\renewcommand*{\Affilfont}{\normalsize\normalfont}
\renewcommand*{\Authfont}{\bfseries}    % make author names boldface    
\setlength{\affilsep}{2em}   % set the space between author and affiliation

\newsavebox\affbox


\begin{document}

\begin{center}
  \includegraphics[width=0.3\textwidth]{imagens-template/ucsal_logo.png}
\end{center}
{\let\newpage\relax\maketitle}

\clearpage
\pagenumbering{arabic} % Retoma a numeração normal
\begin{resumoumacoluna}
  A Engenharia de Software é de extrema importância para possibilitar a modernização do setor público, uma vez que a área faz com que serviços sejam acessibilizados por meio da digitalização, promovendo a eficiência e transparência. Por outro lado, problemas como a burocracia e a inexistência de diretrizes tornam a modernização do setor uma tarefa desafiadora. As iniciativas como o INSS Digital, do Gov.br e o Meu SUS digital são apenas alguns exemplos que comprovam como a inovação tecnológica é benéfica. No entanto, a fim de alcançar o desenvolvimento completo, políticas claras devem ser estabelecidas, e o investimento em digitalização deve ser contínuo para tornar a gestão pública mais acessível e ágil.

  \vspace{\onelineskip}

  \noindent
  \textbf{Palavras-chaves}: Transformação digital, software, inovação.
\end{resumoumacoluna}

\clearpage

\textual

% Apresentação do Tema, Contextualização 
\section{Introdução}
A transição do mundo analógico para o digital fez com que a sociedade incorporasse o uso de tecnologias em seu cotidiano, inclusive para acessar serviços governamentais. Diante disso, o Estado busca adaptar-se a essa realidade, promovendo a digitalização e a simplificação do acesso aos seus serviços \cite{viana2021transformaccao}.
A Engenharia de Software é uma profissão que tem um papel fundamental nesse contexto, pois é catalisadora da transformação tecnológica, portanto ela é o agente responsável pela adaptação dos serviços governamentais para o mundo digital. No entanto, essa transformação não ocorre sem desafios, como o burocratizado sistema governamental brasileiro e seus processos lentos e complexos, além da necessidade de proteção da privacidade dos cidadãos e a garantia do uso ético da tecnologia por parte do Estado. Por outro lado, são proporcionadas diversas oportunidades, tais como a redução de custos operacionais e eficiência nos programas governamentais, gestão pública moderna, inclusão digital e acessibilidade. Diante destes desafios e oportunidades, como a Engenharia de Software pode contribuir para o desenvolvimento do Estado brasileiro?

\section{O mundo moderno não poderia existir sem o software}
Infraestruturas e serviços nacionais são controlados por sistemas computacionais, e a maioria dos produtos elétricos inclui um computador e um software que o controla. A manufatura e a distribuição industriais são totalmente informatizadas, assim como o sistema financeiro. A área de entretenimento, incluindo a indústria da música, jogos de computador, cinema e televisão, faz uso intensivo de software. Nos últimos anos, percebeu-se o aumento da velocidade das mudanças sociais, econômicas e tecnológicas [...] e a necessidade de inovação constante.\cite{albertin2021transformaccao}

Portanto, a Engenharia de Software é essencial para a inovação. Tal disciplina transcende a simples codificação, abrangendo um ecossistema complexo de práticas, metodologias e ferramentas que visam garantir a eficiência, eficácia e qualidade no desenvolvimento de sistemas de software. A Engenharia de Software tem por objetivo apoiar o desenvolvimento profissional de software, mais do que a programação individual. Ela inclui técnicas que apoiam especificação, projeto e evolução de programas. Seu papel é fundamental na inovação tecnológica, atuando como espinha dorsal para o desenvolvimento de produtos e serviços que transformam setores inteiros. Por exemplo, "Inteligência Artificial (IA), internet das coisas (IoT), aprendizado de máquinas (ML) são inovações que chegaram e ficaram para sempre na sociedade" \cite{marques2023ciencia}, sua integração em sistemas existentes tem permitido a criação de soluções mais inteligentes e conectadas. Essas tecnologias não apenas melhoram a funcionalidade dos produtos, mas também abrem novas oportunidades de negócios e experiências para novos usuários.

\section{Inovação tecnológica no serviço público}
A transformação digital no setor público é uma necessidade cada vez maior para a eficiência e a segurança no uso de dados. Esse conceito está relacionado à digitalização de documentos, ferramentas e processos para revolucionar a produtividade. No Brasil, esse avanço tecnológico tem sido impulsionado por diversas políticas governamentais. Um exemplo é o Programa Nacional de Desburocratização, implementado na década de 1990, que buscava modernizar a administração pública por meio da digitalização de processos administrativos \cite{grin2015programa}. Outro exemplo é a A Lei Geral de Proteção de Dados Pessoais (LGPD), Lei n° 13.709/2018, que estabelece regras para a coleta, armazenamento e compartilhamento de informações pessoais que é essencial para a proteção das infraestruturas críticas do governo, prevenindo ataques cibernéticos e garantindo a integridade dos dados, visando garantir a privacidade e a segurança dos cidadãos\cite{govbr_mds_lgpd}.
 

Entretanto, a burocracia lenta para a transformação digital no setor público perdura e se torna um empecilho que tem como consequências a dificuldade e demora nos processos, aumentando a complexidade de integração e diminuindo a disponibilidade de recursos. Uma das maiores causas dessa burocracia exacerbada está na ausência de políticas e diretrizes claras e isso dificultou a modernização dos processos administrativos e a falta de um marco normativo consistente agravou os entraves burocráticos e retardou a transformação digital no setor público. Além da falta de continuidade nas políticas públicas e a extinção de órgãos responsáveis pela modernização administrativa, como o Ministério da Administração Federal e Reforma do Estado (MARE) em 1 de janeiro de 1999 \cite{grin2015programa}.

\subsection{O engenheiro de software como agente da inovação}

Uma abordagem para estimular a inovação no setor público é o estabelecimento de programas de apoio a startups e empresas de base tecnológica, com o objetivo de promover o desenvolvimento de soluções inovadoras voltadas para a administração pública. É nestes lugares onde está o engenheiro de software, que pode contribuir para a criação de sistemas eficientes e seguros, capazes de otimizar processos e melhorar a prestação de serviços públicos. 

\subsection{Exemplos de inovação tecnológica no serviço público}

A digitalização do serviço do INSS, o Projeto \textit{INSS Digital} começou a ser implementado em 2017 e propôs a aplicação da tecnologia da informação de forma a viabilizar uma externalização de parte das atividades do INSS, ou seja, o requerimento de benefícios previdenciários passou a ser feito de forma digital através de entidades parceiras que estão em contato direto com o INSS. Antes da implementação desse projeto, o acesso ao INSS era marcado por deficiências na prestação do serviço e no atendimento ao público \cite{pinheiro2021transformaccao}.

Meu SUS Digital, aplicativo oficial do Ministério da Saúde, que visa ser uma porta de fácil acesso para os serviços do SUS [Sistema Único de Saúde], juntando, em um só portal, as partes mais importantes do serviço. Alguns dessas incluem: histórico de vacinação, quais vacinas foram tomadas e quais faltam; permite identificar estabelecimentos de saúde próximos à sua localização, de acordo com o tipo de serviço desejado; Aplicações (Uma área com diversos aplicativos de saúde que fazem parte do dia a dia do cidadão, como o Peso saudável, Transplantes, Saúde População Negra, entre outros.) \cite{meususdigital}

O Portal Gov.br criado em parceria do Ministério da Gestão e da Inovação em Serviços Públicos com o Serviço Federal de Processamento de Dados [Serpro], é um projeto que unifica os canais digitais do Estado, facilitando o acesso do cidadão a seus documentos e informações sobre as áreas do governo. O portal oferece serviços como: acesso a informações sobre benefícios sociais, consulta de processos, emissão de documentos, entre outros, economizando tempo e evitando deslocamentos desnecessários \cite{govbr}.

O SISU (Sistema de Seleção Unificado) é um sistema eletrônico gerido pelo Ministério da educação, cujo intuito é oferecer vagas de graduação em faculdades públicas para os participantes do Exame Nacional do Ensino Médio [ENEM]. Facilita a gestão de participantes para o curso superior, enquanto atualiza os candidatos sobre sua posição e suas chances de conseguir a vaga. \cite{sisuUFC2025}

\section{Conclusão}
Dessa forma, são notórias a presença e a importância da engenharia de software para o governo, para os cidadãos, e para a sociedade em um âmbito geral, vendo que, não apenas são introduzidas novas formas de ajudar a população em suas questões, que vão desde as básicas até as mais complexas, mas também organiza e facilita o recebimento e a compreensão das informações pelo Estado criando uma relação mais transparente e coesa. 

Em conclusão, a Engenharia de Software é fundamental para o funcionamento e a evolução da sociedade moderna, impulsionando a inovação e a transformação digital em diversos setores. No entanto, no setor público, a burocracia e a falta de uma administração concisa dificultam a implementação eficiente de soluções tecnológicas. Para superar esses desafios, é essencial estabelecer diretrizes claras e promover a inovação. A adoção de práticas de Engenharia de Software pode otimizar processos, aumentar a produtividade e garantir a segurança no uso de dados. Assim, a tecnologia pode ser plenamente aproveitada para melhorar serviços e beneficiar a sociedade como um todo.


% Formatação da bibliografia
%bibliographystyle{plain}
\bibliography{referencias} % Assume que você tem um arquivo referencias.bib

\end{document}
